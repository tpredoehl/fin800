\documentclass[]{article}
\usepackage{amsmath}
\usepackage{amssymb}

%opening
\title{FIN800 Discrete Time Financial Economics - Homework 1}
\author{Giovanni Maffei, Timo Predöhl}

\begin{document}

\maketitle

\section{portfolio choice}
An agent with initial wealth $w_t$ at time t invests a fraction of wealth $\frac{\phi}{w_t}$ in the risky asset and the remainder $(1-\frac{\phi}{w_t})$ in the riskless asset. The resulting wealth at time t+1 is:
\begin{align*}
	R_{f,t+1} &= R_{f} = 1+r_f = const.\\
	R_{t+1} &= 1 + \mu + \beta x_t + \sigma_r \eta_{r, t+1}\\
	w_{t+1} &= w_t[(1-\frac{\phi}{w_t})R_f+\frac{\phi}{w_t} R_{t+1}]\\
	&= w_t[R_f - \frac{\phi}{w_t} R_f + \frac{\phi}{w_t} R_{t+1}]\\
	&= w_t[R_f + \frac{\phi}{w_t} (R_{t+1} - R_f)]\\
	&= w_t R_f + \phi (R_{t+1} - R_f)
\end{align*}

The agent's utility function then is:
\begin{align*}
	u(w_{t+1}) &= -e^{-aw_{t+1} - bESG_t}\\
	&= -e^{-a(w_t R_f + \phi (R_{t+1} - R_f)) - bESG_t}\\
\end{align*}

The random variable $R_{t+1}$ is normally distributed with 
\begin{align*}
	E[R_{t+1}] &= 1+\mu+\beta x_t\\
	V[R_{t+1}] &= E[(R_{t+1} - E[R_{t+1}])^2]\\
	&= E[(1 + \mu + \beta x_t + \sigma_r \eta_{r, t+1} - 1 - \mu - \beta x_t)^2]\\
	&= E[\sigma_r^2 \eta_{r, t+1}^2]\\
	&= \sigma_r^2 E[\eta_{r, t+1}^2]\\
	&= \sigma_r^2
\end{align*}

The random variable $\tilde{x} = -a(w_t R_f + \phi (R_{t+1} - R_f)) - bESG_t$ is normally distributed with 
\begin{align*}
	E[\tilde{x}] &= -aw_t R_f -a \phi (1+\mu+\beta x_t - R_f)) - bESG_t\\
	&= -aw_t R_f - a \phi - a \phi \mu - a \phi \beta x_t + a \phi R_f - bESG_t\\
	V[\tilde{x}] &= E[(\tilde{x} - E[\tilde{x}])^2]\\
	&= E[(-aw_t R_f - a \phi - a \phi \mu - a \phi \beta x_t - a \phi \sigma_r \eta_{r, t+1} + a \phi R_f - bESG_t \\
	&+ aw_t R_f + a \phi + a \phi \mu + a \phi \beta x_t - a \phi R_f + bESG_t)^2]\\
	&= E[(- a \phi \sigma_r \eta_{r, t+1})^2]\\
	&= E[a^2 \phi^2 \sigma_r^2 \eta_{r, t+1}^2]\\
	&= a^2 \phi^2 \sigma_r^2 E[\eta_{r, t+1}^2]\\
	&= a^2 \phi^2 \sigma_r^2
\end{align*}

Therefore, using the fact that the expectation of the exponential of a normally distributed random variable is the exponential of the mean plus half of the variance, the expected utility is
\begin{align*}
	E[u(w_{t+1})] &= E[-e^{-a(w_t R_f + \phi (R_{t+1} - R_f)) - bESG_t}]\\
	&= -e^{-aw_t R_f -a \phi (1+\mu+\beta x_t - R_f) - bESG_t + \frac{1}{2}a^2 \phi^2 \sigma_r^2}
\end{align*}

Maximizing expected utility is equivalent to maximizing the term $$-aw_t R_f -a \phi (E[R_{t+1}] - R_f) - bESG_t + \frac{1}{2}a^2 \phi^2 \sigma_r^2$$.

Differentiating with respect to $\phi$ and setting the derivative to zero yields: 
\begin{align*}
	0 &= \frac{\partial (-aw_t R_f -a \phi (E[R_{t+1}] - R_f) - bESG_t + \frac{1}{2}a^2 \phi^2 \sigma_r^2)}{\partial(\phi)}\\
	&= -a (E[R_{t+1}] - R_f) + \phi a^2 \sigma_r^2\\
	\phi &= \frac{a (E[R_{t+1}] - R_f)}{a^2 \sigma_r^2}\\
	&= \frac{(E[R_{t+1}] - R_f)}{a \sigma_r^2}
\end{align*}

\subsection{What will be the optimal portfolio choice of the investor ?}
The optimal amount $\phi$ to invest in the risky asset is $$\phi^* = \frac{(E[R_{t+1}] - R_f)}{a \sigma_r^2}$$.

The optimal choice $\phi^*$ 
\begin{itemize}
	\item is an increasing function of the risk premium $E[R_{t+1}]-R_f$,
	\item is a decreasing function of the variance $\sigma_r^2$,
	\item is a decreasing function of the investor’s absolute risk aversion a,
	\item is positive when the risk premium is positive, and
	\item does not depend on the initial wealth $w_t$.
\end{itemize}.

\subsection{What will be the mean and variance of the corresponding optimal wealth?}
Optimal wealth is obtained by inserting the optimal portfolio into the wealth function:
\begin{align*}
	E[w_{t+1}] &= w_t R_f + \phi^* (E[R_{t+1}]-R_f)\\
	&= w_t R_f + \frac{(E[R_{t+1}] - R_f)^2}{a \sigma_r^2}\\
	&= w_t R_f + \text{Sharpe}^2
\end{align*}

\subsection{Compute the welfare of the investor.}
The welfare of the investor is obtained by inserting the optimal portfolio $\phi^*$ into the term

\subsection{What is the distribution of the optimal wealth ? Comment.}
\subsection{Comment on the impact of preference for ESG on the optimal portfolio.}
\subsection{Now assume that the ESG rating is uncertain and follows a normal distribution ($\mu, \sigma$), independent from $\eta_{r,t+1}$. Repeat the previous questions.}

\section{inter-temporal consumption}

\end{document}
