\documentclass[12pt]{article}
\usepackage{amsmath}
\usepackage{amssymb}
\usepackage{setspace}
\usepackage{indentfirst}
\usepackage{parskip}

%\usepackage{easyReview} %used for annotation. Cannot be run together with (changes) - command "highlight" defined twice
\usepackage{changes}%use while actively annotating

%\usepackage[final]{changes} % once we are happy with all annotation, we can accept all proposed changes and show only the final document.

% simply select text you want to annotate and start typing the command \replacedGM{the selected text will appear here}{and here you enter the new replacement text}
\definechangesauthor[name={Timo}, color=orange]{tp}
\definechangesauthor[name={Giovanni}, color=green]{gm}
\newcommand{\replacedTP}[2]{\replaced[id=tp]{#2}{#1}}
\newcommand{\addedTP}[1]{\added[id=tp]{#1}}
\newcommand{\deletedTP}[1]{\deleted[id=tp]{#1}}
\newcommand{\replacedGM}[2]{\replaced[id=gm]{#2}{#1}}
\newcommand{\addedGM}[1]{\added[id=gm]{#1}}
\newcommand{\deletedGM}[1]{\deleted[id=gm]{#1}}


%opening
\title{FIN800 Discrete Time Financial Economics - Homework 1}
\author{Giovanni Maffei, Timo Predöhl}

\begin{document}
	
	\maketitle
	
	%\listofchanges[]
	% I also replace all r_{t,t+1} with r_{t+1} and r_{f,t+1} with r_f to improve legibility
	
	
	\begin{spacing}{1}
		\tableofcontents		
	\end{spacing}
	
	\pagebreak
	
	\section{Exercise 1}
	\subsection{Set-up and Definitions}
	\begin{itemize}
		\item N=2 assets, a riskless asset and a risky \replacedTP{one}{asset}
		\item Frictionless markets
		\item \replaced[id=tp]{1-period model, starting at t, ending at t+1}{Two periods: t = today and t+1 = end of the investment period }
		\item The riskless asset returns $r_f = r$ (deterministic and known in t) 
		\item $\eta_{r, t+1} \sim N(0,1)$
		\item The risky asset returns $r_{t+1} = \mu + \beta x_t + \sigma_r \eta_{r, t+1}$
		\item $E_{t}[r_{t+1}] = \mu + \beta x_t$
		\item $V_{t}[r_{t+1}] = \sigma_r^2$
		\item $W_{t} = $ investable wealth (endowment at time t \replacedTP{minus consumption $c_t$}{after consumption $c_t$ took place})
		\item  $U_{t}(W_{t+1}) = -e^{-aW_{t+1}-bESG_t}$
		\item $a > 0 $ (the investor has positive absolute risk aversion)
		\item The investor knows the distribution of the return, the underlying parameters and the ESG score of the risky asset.
	\end{itemize}
	\subsection{What will be the optimal portfolio choice of the investor ?}
	\replacedTP{Noting: (i) the investor has already decided how much to consume at time t ($c_t$) and (ii) the marginal utility of consumption is positive ($u'>0$)}{Having consumed $c_t$}, the agent \replacedTP{needs to allocate its investable wealth }{allocates its remaining investable wealth $W_t$} across the only two assets available in the investable universe: (i) the risk-free bond and (ii) the risky asset. The share of the available wealth attributed to the risky asset is defined as $\theta_t$. \\
	The resulting wealth at t+1 \replacedTP{becomes}{is}:
	\begin{align*}
		W_{t+1} &= (W_t)\left[(1-\theta_t)(1+r_f)+ \theta_t(1+r_{t+1}) \right]\\
		&= (W_t) \left[1 -\theta_t + r_f - \theta_t r_f + \theta_t + \theta_t r_{t+1}\right]\\
		&= (W_t) \left[1 + r_f + \theta_t(r_{t+1}-r_f) \right]
	\end{align*} \vspace{-1.75em}
	
	The optimal portfolio choice is found by maximizing the expected utility in t+1 such that (by replacing $W_{t+1}$ with the equation above):
	\begin{align*}
		\max_{(\theta_t)} E_t[U_{t}(W_{t+1})] &= E_t[-e^{(-aW_t) [1 + r_f + \theta_t(r_{t+1}-r_f)] - bESG_t}] \\
	\end{align*}\vspace{-3.65em}

	\replacedTP{Additionally, as $r_{t+1}$ is normally distributed, we can observe that the argument of the exponential function is a linear combination of a random variable normally distributed. Hence, considering that the expectation of the exponential of a normal corresponds to the exponential of the random variable's mean plus half of the variance, it follows that the optimization problem can be rewritten as:}{The exponential is normally distributed given $r_{t+1}$ is normally distributed. Since the expectation of a normal exponential equals the exponential of its mean plus one half of its variance:}\vspace{-3em}
	
	\begin{align*}
		\max_{(\theta_t)} E_t[U_{t}(W_{t+1})] &= -e^{(-aW_t) [1 + r_f + \theta_t(E_{t}[r_{t+1}]-r_f)] - bESG_t + \frac{1}{2}a^2W_t^2\theta_t^2\sigma_r^2} \\
	\end{align*} \vspace{-3.25em}
	
	Given the investor has a utility function \replacedTP{chartacterized}{characterized} by non-satiation and risk aversion ($u'>0$ and $u''<0$), \addedTP{optimal} portfolio weights are \replacedTP{defined by optimizing the expression below (the argument of the exponential)}{found by maximizing the exponential}: \vspace{-3em}\\
	
	\begin{align*}
		(-aW_t) [1 + r_f + \theta_t(E_{t}[r_{t+1}]-r_f)] - bESG_t + \frac{1}{2}(a^2W_t^2\theta_t^2\sigma_r^2) \\
	\end{align*} \vspace{-3em}
	
	The first-order condition (FOC) \addedTP{with respect to $\theta_t$ }is: \vspace{-2.5em}\\
	
	\begin{align*}
		0 &= (-aW_t)(E_{t}[r_{t+1}]-r_f)) + (a^2W_t^2\theta_t\sigma_r^2) \\
	\end{align*} \vspace{-3em}
	
	The optimal portfolio that solves the FOC is: 
	
	\begin{equation} \label{Optimal weight}
		\theta_t^* = \frac{1}{aW_t}\frac{(E_{t}[r_{t+1}]-r_f)}{\sigma_r^2} = \frac{1}{aW_t\sigma_r}Sharpe\\
	\end{equation}
	
	\replacedTP{A few observations on}{Observations on optimal portfolio weights, given $U(W_{t+1})$} (\ref{Optimal weight}):
	\begin{itemize}
		\item \addedTP{$\frac{\partial \theta_t}{\partial a} < 0$ allocation to risky assets decreases in $a$. Risk-averse investors prefer bonds over risky assets.} 
		\item \addedTP{$\frac{\partial \theta_t}{\partial W_t} < 0$ allocation to risky assets decreases in $W_t$. Wealthy investors prefer bonds over risky assets.} 
		\item \addedTP{$\frac{\partial \theta_t}{\partial \sigma_r} < 0$ allocation to risky assets decreases in $\sigma_r$. Investors prefer risky assets with lower total risk.}
		\item \addedTP{$\frac{\partial \theta_t}{\partial \text{Sharpe}} > 0$ allocation to risky assets increases in the Sharpe ratio. Any investor prefers higher risk-adjusted excess returns.}
		\item \addedTP{$\theta_t > 0$ given $a, W_t, \sigma_r$ and Sharpe $> 0$. Any investor will keep some holdings of the risky asset in the optimal portfolio.}
		\item The risky-asset optimal portfolio weight depends on $aW_t$ which corresponds to the relative risk aversion of the investor\footnote {We can also observe that this solution is not scale invariant: as the stock of investable wealth grows, the investment in the risky asset remains constant implying IRRA (Increasing Relative Risk Aversion) preferences.}.\todo{discuss footnote IRRA}
		\item The second term of \ref{Optimal weight} corresponds to the Sharpe ratio ($Sharpe$) of the risky asset: the expected excess return of the risky asset standardized by its volatility. Such metric, further divided by the asset standard deviation in \ref{Optimal weight}, measures how efficiently the risky asset rewards \textit{ex-ante} the quadratic risk borne by the investor. \todo{delete?}
		\item Assuming $E_{t}[r_{t+1}] - r_f$ is positive the Reader can infere that, regardless of the actual preferences of the subject, any investor will keep some holdings of the risky asset in the optimal portfolio.\todo{delete?}\\ \vspace{-1.5em}    
	\end{itemize}
	
	\subsection{What will be the mean and variance of the corresponding optimal wealth ?} \label{Optimal Wealth}
	
	Given (\ref{Optimal weight}), we can rewrite the optimal \addedTP{wealth} $W_{t+1}^{*}$ as: \vspace{-2.5em}\\
	
	\begin{align*}
		W_{t+1}^{*} &= (W_t) \left[1 + r_f + \theta_t^*(r_{t+1}-r_f) \right]\\
		&= (W_t) \left[1 + r_f + \frac{1}{(aW_t)}\frac{(E_{t}[r_{t+1}]-r_f)}{\sigma_r^2}(r_{t+1}-r_f) \right]\\
		&= (W_t)(1 + r_f) + \frac{1}{a}\frac{(E_{t}[r_{t+1}]-r_f)}{\sigma_r^2}(r_{t+1}-r_f)
	\end{align*} \vspace{-1.35em}
	
	The mean of $W_{t+1}^{*}$ becomes:
	\begin{align*}
		E_{t}[W_{t+1}^{*}] = E_{t}[(W_t)(1 + r_f)] + E_{t}[\frac{1}{a}\frac{E_{t}[r_{t+1}]-r_f}{\sigma_r^2}(r_{t+1}-r_f)]
	\end{align*} \vspace{-1.25em}
	
	\replacedTP{	By noting that: i) $W_t$ and ii) $r_f$ are known in t and the expectation (conditional to t) of the expected return of the risky asset corresponds to the expected return itself by the law of iterated expectations, we can find that:   }{Given $E_t[W_t]=W_t$, $r_f=\text{const.}$ and $E_t[E_t[r_{t+1}]] = E_t[r_{t+1}]$ by LIE:} 
	\begin{align} \label{Mean Wealth}
		E_{t}[W_{t+1}^{*}] &= W_t(1 + r_f) + \frac{1}{a}\frac{(E_{t}[r_{t+1}]-r_f)^2}{\sigma_r^2} \notag\\
		&= W_t(1 + r_f) + \frac{1}{a}Sharpe^2 \\ \notag
	\end{align} \vspace{-3.25em}  
	
	\replacedTP{A few o}{O}bservations on (\ref{Mean Wealth}):
	\begin{itemize}
		\item The expected wealth in t+1 \replacedTP{corresponds to}{is floored at} the investable wealth in t capitalized at the risk free rate plus \addedTP{a risky bet equal to }the square of the sharpe ratio of the asset divided by the absolute risk aversion of the investor.
		\item The lower the absolute risk aversion and the higher the expected wealth of the investor is. This result arises from the consideration that the investor will tilt her portfolio to the risky asset clipping a positive reward for the risk borne hence, improving the expectation of her wealth at the end of the investment period.       
	\end{itemize}
	
	The variance of $W_{t+1}^{*}$ \replacedTP{becomes}{is}: \vspace{-0.5em}
	\begin{align*}
		V_{t}(W_{t+1}^{*}) = V_{t}[W_t(1 + r_f) + W_t[\frac{1}{a}\frac{(E_{t}[r_{t+1}]-r_f)}{\sigma_r^{2}}(r_{t+1}-r_f)]]
	\end{align*} \vspace{-1.75em}
	
	\replacedTP{By observing that: i) $W_t$, ii) $r_f$ and iii) $E_{t}[r_{t+1}]$ are determined in t and r}{R}ecalling that $V(ax + b) = a^2V(x)$ $\forall$ a,b constant, we derive: \vspace{-1.25em}
	
	\begin{align} %\label{Variance Wealth}
		V_{t}(W_{t+1}^{*}) &= (W_t^2)[\frac{1}{(W_t^2a^2)}\frac{(E_{t}[r_{t+1}]-r_f)^2}{\sigma_r^4}\sigma_r^2] \notag \\
		&=\frac{1}{a^2}\frac{(E_{t}[r_{t+1}]-r_f]^2}{\sigma_r^2} \notag\\
		&= \frac{1}{a^2}Sharpe^2 \\	\notag
	\end{align}  \vspace{-3em}
	
	The variance of the optimal wealth corresponds to the squared sharpe ratio of the risky asset standardized by the quadrate of the absolute risk aversion of the investor. \addedTP{Given $\frac{\partial V_t (W_{t+1}^*)}{\partial a} < 0$ directly implies that less risk-averse investors tolerate less variance in expected future wealth.}
	
	\subsection{Compute the welfare of the investor}
	Following the observation that the expectation of the exponential of a normally distributed random variable corresponds to the exponential of the mean plus half of the variance and grouping the argument of the exponential by the absolute risk aversion of the investor, $a$, we find that: \vspace{-1.5em}
	
	\begin{align*}
		E_t[U_{t}(W_{t+1}^{*})] &= -e^{(-aW_t) [1 + r_f + \theta_t^*(E_{t}[r_{t+1}]-r_f)] - bESG_t + \frac{1}{2}a^2W_t^2(\theta_t^{*})^2\sigma_r^2} \\
		&= -e^{-a [W_t(1 + r_f) + W_t(\theta_t^*(E_{t}[r_{t+1}]-r_f)) + \frac{b}{a}ESG_t - \frac{1}{2}aW_t^2(\theta_t^{*})^2\sigma_r^2]}
	\end{align*} \vspace{-2em}
	
	The argument of the exponential within square brackets represents the certainty equivalent ($CE$) of the random wealth $W_{t+1}^{*}$ and after substituting for $\theta_t^{*}$: \vspace{-2em} \todo{discuss: $W_t$ not a f(ESG)!}
	
	\begin{align*}
		CE^* &= [W_t(1 + r_f) + W_t(\theta_t^*(E_{t}[r_{t+1}]-r_f)) + \frac{b}{a}(ESG_t) - \frac{1}{2}(aW_t^2(\theta_t^{*})^2\sigma_r^2)] \\
		&= [W_t(1 + r_f) + \frac{1}{a}\frac{(E_{t}[r_{t+1}]-r_f)^2}{\sigma_r^2} + \frac{b}{a}(ESG_t) - \frac{1}{2a}\frac{(E_{t}[r_{t+1}]-r_f)^2}{\sigma_r^2}]
	\end{align*} \vspace{-1.5em}
	
	Simplyfing further the terms, the expression becomes: \vspace{-1.75em}
	
	\begin{align}
		CE^* = [W_t(1 + r_f) + \frac{1}{2a}Sharpe^2 + \frac{b}{a}(ESG_t)] 
	\end{align} 

	\begin{itemize}
		\item \addedTP{$\frac{\partial CE^*}{\partial a}<0$, lower risk-aversion increases the CE or willingness to take a risky bet}
		\item \addedTP{$\frac{\partial CE^*}{\partial bESG_t}>0$, higher ESG rating increases the CE or willingness to take a risky bet}
	\end{itemize}
	
	\subsection{What is the distribution of the optimal wealth ? Comment}
	We can observe that the distribution of $W_{t+1}^{*}$ is normal with mean and variance as per \ref{Optimal Wealth}. Indeed, from the equation below, we see that $W_{t+1}^{*}$ is \replacedTP{a linear combination of a}{afine transformation of the} normal random variable ($r_{t+1}$)\footnote {$\theta_t^*$ is entirely determined in t.}.
	\begin{align*}
		W_{t+1}^{*} = (W_t) \left[1 + r_f + \theta_t^*(r_{t+1}-r_f) \right]
	\end{align*} \vspace{-2.5em}
	
	\subsection{Comment on the impact of preference for ESG on the optimal portfolio?}
	\todo{consider an alternative}\addedTP{While the optimal portfolio $\theta_t^*$ (\ref{Optimal weight}) is indifferent to the ESG rating, the expected utility of future wealth considers the preference (with $b>0$) for those risky assets that are better rated in terms of ESG. Given that $bESG_t$ is determined before optimizing the portfolio, the investor would in fact perform a two-stage optimization program. First, a minimum ESG rating is determined according to preference b. The universe of investable risky assets is then filtered to those acceptable asset, which are at or above the minimum ESG rating. Secondly, the optimal portfolio $\theta_t^*$ of acceptable risky assets is formed. Considering two agents with preferences $b_1$ for very high and $b_2$ for very low or indifferent to ESG ratings, it is obvious that the investable universe of the agent with preference $b_1$ for very high ESG rating must be smaller. Given a smaller universe of investable assets, $\theta_{t,1}^*$ will result in a lower Sharpe ratio and consequently lower expected future wealth. However, this agent receives additional and higher utility, given $b_1 > b_2$ from $b_1ESG_t$, which may compensate for the lower expected future monetary wealth. Given suitable preferences, both agents may end up with equal expected utility.}
	The impact of preference for ESG is expressed in the utility function as a component of the argument of the exponential: $bESG_t$. 
	Assuming the investor has a preference for ESG-friendly securities, the term $b$ will be $> 0$ and can be interpreted as the non-monetary benefit arising from holding the risky asset in opposition to the risk-free rate (neutral to the ESG profile of the overall portfolio)\footnote {Vice versa for the following instances: i) $b < 0 $ in case of a penalty for ESG-friendly choises or ii) a non ESG-friendly stock with a negative ESG impact. Such cases have not been considered as rather counterintuitive.}. Therefore, it follows that $b$ emerges as an absolute risk aversion coefficient to the ESG factor:
	
	\begin{itemize}
		\item The impact for ESG preferences is just a mere (positive) scaling factor of the utility function and affects all investors in the same way. Indeed, we showed with \ref{Optimal weight} that each investor, regardless of her preferences, has positive holdings of the risky asset and therefore, will capture the benefit from being exposed to a favourably ESG-rated company. Additionally, it is also worth noting that $ESG_t$ is known at the beginning of the period  henceforth it doesn't impact the solution of the optimization problem.   
		\item The fact that all investors share the same benefit derives from appreciating that $ESG_t$ only captures the ESG rating of the risky asset without being modulated for the actual wealth invested.
		\item By considering the ESG rating of the asset, we can also notice that the argument of the utility function is not anymore just the wealth of the investor but it is a \textit{mixture} of money and ESG rating. It follows that the investor will need to focus on the relative preference of these two terms $b/a$, with the ESG component becoming more and more significant as $b$ grows. \vspace{-1em} 
	\end{itemize}
	
	\subsection{Assumption of ESG rating stochasticity}
	Assuming: i) $ESG_t \sim N(\mu_{ESG},\sigma_{ESG})$ and that ii) $\rho(ESG_t,\eta_{r, t+1}) = 0 $ due to independence, we can observe that:
	\begin{itemize}
		\item There are now two stochastic terms in the utility function of the investor: $ESG_t$ and $r_{t+1}$ (the latter embedded in $W_{t+1}$).\\ Therefore, the mean and the variance of the argument: $(-aW_t) [1 + r_f + \theta_t(r_{t+1}-r_f)] - bESG_t$ are
		\subitem Mean: $(-aW_t) [1 + r_f + \theta_t(E[r_{t+1}]-r_f)] - b\mu_{ESG}$
		\subitem Variance: $a^2W_t^2\theta_t^2\sigma_r^2 + b^2\sigma_{ESG}^2$
		\item As the impact of the preference for ESG does not depend on the actual amount of wealth invested in the asset and there is no correlation with $r_{t+1}$, we would expect such uncertainty to emerge as a non-hedgeable risk which will not impact the optimal portfolio choice\footnote {It is rather modelled as a switch variable such that the investor perceives the non-monetary benefit of investing in an ESG-friendly asset just by having exposure to it regardless of how exposed the portfolio is. As a next step, it would be interesting to model: i) the dependence of $bESG_t$ on $\theta$ and ii) the absence of independence between $\eta_{r, t+1}$ and $ESG_t$.}.
		\item Although the ESG rating is ordinal in nature and might be expressed as an alphanumeric metric, in this case the assumption of normality implies that $ESG_t$ covers the interval: $(-\infty,+\infty )$.
	\end{itemize}
	\subsubsection{Optimal portfolio choice}
	Following the original setup of 1.2, we can see now that the argument of the exponential is still a normal random variable. Indeed, such variable is represented as a linear combination of two normals: $W_{t+1}$\footnote{$W_{t+1}$ itself is a linear combination of the normal: $r_{t+1}$.} and $ESG_t$. Henceforth we have: 
	\begin{align*}
		\max_{(\theta_t)} E_t[U_{t}(W_{t+1})] &= E_t[-e^{(-aW_t) [1 + r_f + \theta_t(r_{t+1}-r_f)] - bESG_t}] \\
		&= -e^{(-aW_t) [1 + r_f + \theta_t(r_{t+1}-r_f)] - b\mu_{ESG} +\frac{1}{2}a^2W_t^2\theta_t^2\sigma_r^2 + \frac{1}{2}b^2\sigma_{ESG}^2}]
	\end{align*} \vspace{-2em}
	
	As before, the solution is obtained by maximizing the argument of the above function: \vspace{-0.5em}
	\begin{align*}
		(-aW_t) [1 + r_f + \theta_t(r_{t+1}-r_f)] - b\mu_{ESG} +\frac{1}{2}a^2W_t^2\theta_t^2\sigma_r^2 + \frac{1}{2}b^2\sigma_{ESG}^2
	\end{align*}
	The FOC becomes:
	\begin{align*}
		0 = (-aW_t)(E_{t}[r_{t+1}]-r_f)) + (a^2W_t^2\theta_t\sigma_r^2)
	\end{align*} \vspace{-1.75em}
	
	It follows that \deletedTP{there is no difference from the optimal solution, }$\theta_{t}^*$, shown in 1.2 \addedTP{may remain the optimal solution, but it is unclear if the investable universe is the same. In the deterministic case, the agent chooses a minimum level of acceptable ESG ratings, derives a suitable investable universe and optimizes the portfolio on that basis. If uncertainty enters in relation to ESG ratings, then the agent will still have to form an opinion on how to define her investable universe. The two universes may differ and consequently $\theta_t^*$ may differ as well.} 
	\subsubsection{Mean and variance of optimal wealth}
	As a consequence of the above result, we can infer that the mean, the variance and the distribution of the optimal wealth \replacedTP{will be unchanged}{remains functionally the same, but may differ numerically given possibly different choosen investable universes} .\todo{separated out}
	\subsubsection{Welfare of the investor}
	
	By grouping for $a$ all the terms of the argument of the exponential in the utility function we find that:
	\begin{align*}
		E_t[U_{t}(W_{t+1}^{*})] &= -e^{(-aW_t) [1 + r_f + \theta_t^*(E_{t}[r_{t+1}]-r_f)] - b\mu_{ESG} + \frac{1}{2}a^2W_t^2(\theta_t^*)^2\sigma_r^2 +\frac{1}{2}b^2\sigma_{ESG}^2 } \\
		&= -e^{-a [W_t(1 + r_f) + W_t(\theta_t^*(E_{t}[r_{t+1}]-r_f)) + \frac{b}{a}\mu_{ESG} - \frac{1}{2}aW_t^2(\theta_t^{*})^2\sigma_r^2 -b\frac{b}{2a}\sigma_{ESG}^2]}
	\end{align*}
	The argument represents the CE of the investor: 
	\begin{align*}
		CE^* &= [W_t(1 + r_f) + W_t(\theta_t^*(E_{t}[r_{t+1}]-r_f)) + \frac{b}{a}\mu_{ESG} - \frac{1}{2}aW_t^2(\theta_t^{*})^2\sigma_r^2 -\frac{b^2}{2a}\sigma_{ESG}^2] \\
		&= [W_t(1 + r_f) + \frac{1}{a}\frac{(E_{t}[r_{t+1}]-r_f)^2}{\sigma_r^2} + \frac{b}{a}\mu_{ESG} - \frac{1}{2a}\frac{(E_{t}[r_{t+1}]-r_f)^2}{\sigma_r^2}-\frac{b^2}{2a}\sigma_{ESG}^2]
	\end{align*}
	\replacedTP{Simplyfing}{Simplifying} the terms and substituting for $\theta_t^{*}$ we have: 
	\begin{align} \label{New CE}
		CE^* = [W_t(1 + r_f) + \frac{1}{2a}Sharpe^2 + \frac{b}{a}\mu_{ESG} -\frac{b^2}{2a}\sigma_{ESG}^2]
	\end{align}
	\deletedTP{The main observation on is that the stochasticity of the ESG factor negatively impacts the certainty equivalent of the investor. Indeed, differently from the non-randomic case, the ESG score of the risky asset is not known in t. Although the risky asset offers on average a favourable ESG exposure , the investor discounts a reduction in the non-monetary benefit perceived in the case that volatility plays adversely.}  \todo{net ESG effect?}
	
	The main observation on \ref{New CE} is that the stochasticity of the ESG factor negatively impacts the certainty equivalent of the investor. Indeed, differently from the non-randomic case, the ESG score of the risky asset is not known in t.\\ Although the risky asset offers on average a favourable ESG exposure\footnote{This is represented by $\frac{b}{a}\mu_{ESG}$ in the CE expression.}, the investor discounts a reduction in the non-monetary benefit perceived in the case that volatility plays adversely.
	
	\addedTP{The addition of the random non-monetary ESG effect would contribute to expected utility of future wealth for as long as $bESG_t > b^2 \sigma_{ESG}^2$.}
	
	\subsubsection{Distribution of optimal wealth}
	\subsubsection{Impact of preference for ESG on optimal portfolio}
	
	As the ESG factor does not correlate with $r_{t+1}$ and it is indifferent from the amount of wealth actually invested in the risky asset, we observed that there is no change from the optimal solution previously found in 1.2. \\ However, the volatility of the factor impacts negatively the CE of the investor transforming the ESG component from $\frac{b}{a}\mu_{ESG}$ to: \vspace{-1.5em}
	
	\begin{align}
		CE^*_{ESG} = \frac{b}{a}\mu_{ESG} -\frac{b^2}{2a}\sigma_{ESG}^2
	\end{align}
	The overall effect of the ESG component depends on the relative importance of $b$ over $a$:
	\begin{itemize}
		\item If ${a\to+\infty}$ and $b$ is finite then we can conclude that the impact of ESG becomes negligible on the utility of the investor.
		\item  If ${b\to+\infty}$ and $a$ is finite, the individual will discount a disproportionally high risk premium (reflected in a lower and lower $CE^*$) due to the uncertainty of the ESG score. However, such risk will emerge as unhedgeable in this model because of:
		\subitem The independence between $ESG_t$ and $\eta_{r, t+1}$ that precludes the factoring of the correlation between the two terms in $\theta_{t}^*$.
		\subitem The absence of scaling on the ESG factor by the portion of the portfolio invested in the risky asset. 
	\end{itemize}
	
	\section{Exercise 2}
	The 2-period, single asset utility maximization program is formulated as
	\begin{align*}
		\max_{c_t, c_{t+1}} \quad & U(c_t, c_{t+1}) = u(c_t) + \delta u(c_{t+1}) = \frac{c_t^{1-\gamma}}{1-\gamma} + \delta \frac{c_{t+1}^{1-\gamma}}{1-\gamma}\\
		\textrm{s.t.} \quad & c_t = y_t - \lambda,\\
		\quad & c_{t+1} = y_{t+1} + \lambda (1+R)\\
	\end{align*}
	
	The budget constraints are reformulated as
	\begin{align*}
		\lambda &= y_t - c_t = \frac{c_{t+1} - y_{t+1}}{1+R}\\
		c_t + \frac{c_{t+1}}{1+R} &= y_t + \frac{y_{t+1}}{1+R}
	\end{align*}
	
	At time t the agent can choose to invest an amount $\lambda$ into a risk-less asset which yields the known return R. Formulating the Lagrangian and FOCs with respect to $c_t$ and $c_{t+1}$ yields
	\begin{align*}
		L &= u(c_t) + \delta u(c_{t+1}) - \phi \left(c_t + \frac{c_{t+1}}{1+R} - y_t - \frac{y_{t+1}}{1+R}\right)\\
		\frac{\partial L}{\partial c_t} &= u_{c_t}' - \phi = 0\\
		\frac{\partial L}{\partial c_{t+1}} &= \delta u_{c_{t+1}}' - \phi \frac{1}{1+R}= 0
	\end{align*}
	
	The inter-temporal marginal rate of substitution (IMRS) describes the rate at which an agent is willing to trade current- for future consumption while maintaining the same level of utility. The inter-temporal elasticity of substitution (IES) is a measure of responsiveness of the growth rate of consumption to the change in the gross interest rate. As interest increase less risk-averse investors will save more by reducing current consumption (substitution effect). At the same time, future income is more heavily discounted resulting in lower total endowment (income effect). The IES describes the resulting net effect.
	
	The FOC yield the inter-temporal marginal rate of substitution (IMRS):
	\begin{align*}
		\frac{\delta u_{c_{t+1}}'}{u_{c_t}'} &= \frac{1}{1+R}\\
		\delta \left( \frac{c_{t+1}}{c_t} \right)^{-\gamma} &= \frac{1}{1+R}\\
		\text{IMRS} &= (1+R)^{-1}
	\end{align*}
	The agent would at time t at most forego and amount equal to $(1+R)^{-1}$ and invest it in the available asset to earn the certain return at time t+1 of $1 = \frac{1}{1+R}(1+R)$. The inter-temporal elasticity of substitution (IES) is derived directly from the IMRS:
	\begin{align*}
		\frac{c_{t+1}}{c_t} &= \delta^{\frac{1}{\gamma}}(1+R)^{\frac{1}{\gamma}}\\
		\ln \left(\frac{c_{t+1}}{c_t}\right) &= \frac{1}{\gamma} \ln \delta + \frac{1}{\gamma} \ln (1+R)\\
		\text{IES} &= \frac{\partial \ln (\frac{c_{t+1}}{c_t})}{\partial\ln (1+R)} = \frac{1}{\gamma}
	\end{align*}
	
	\subsection{What will be the optimal portfolio and how sensitive it is relative to $\gamma$?}
	The optimal portfolio is determined by the agents choice of $c_t$. Given that $\delta$, $\gamma$, the endowment $w_0 = y_t + \frac{y_{t+1}}{1+R} = c_t + \frac{c_{t+1}}{1+R}$ and R are known and certain at time t, a choice of $c_t$ implies a choice of $c_{t+1}$ and thereby the ratio $\frac{c_{t+1}}{c_t}$. The optimal portfolio choice then will be such that $$\frac{c_{t+1}}{c_t} = \delta^{\frac{1}{\gamma}}(1+R)^{\frac{1}{\gamma}}$$
	
	While the sensitivity of the optimal portfolio choice to changes in the interest rate is given by the IES $= \frac{1}{\gamma}$, the sensitivity to changes in risk-aversion $\gamma$ is given by
	\begin{align*}
		\ln \left(\frac{c_{t+1}}{c_t}\right) &= \frac{1}{\gamma} \ln \delta + \frac{1}{\gamma} \ln (1+R)\\
		\frac{\partial \ln \left(\frac{c_{t+1}}{c_t}\right)}{\partial \frac{1}{\gamma}} &= \ln \delta + \ln (1+R)\\
	\end{align*}
	
	A decrease \footnote{Interesting question: What is the direction of causality? Do agents reduce risk aversion, when they observe higher returns or do returns increase when agents happen to have lower risk aversion?} in risk aversion $\gamma$, i.e. an increase of the IES $\frac{1}{\gamma}$ increases the preference for future consumption by a percentage equal to the sum of logs of personal time preference $\delta$ and gross return 1+R. Given constant personal time preference $\delta$, the less risk averse agent accepts to reduce current consumption and invests more in the tradable asset when she is compensated by a corresponding increase in the rate of return. 
	
	\subsection{Rewrite the utility function in a recursive way using a CES aggregator. What will then be the optimal portfolio and how sensitive it is relative to $\gamma$ and the IES ?}
	The original utility function is turned into a CES aggregator using:
	\begin{align*}
		U(c_t, c_{t+1}) &= \frac{1}{1-\gamma} \left[c_t^{1-\frac{1}{\rho}} + \delta c_{t+1}^{1-\frac{1}{\rho}}\right]^{\frac{1-\gamma}{1-\frac{1}{\rho}}}\\
		u_{c_t}' &= \frac{\partial u}{\partial c_t} = \frac{1}{1-\frac{1}{\rho}}\left[c_t^{1-\frac{1}{\rho}} + \delta c_{t+1}^{1-\frac{1}{\rho}}\right]^{\frac{1-\gamma}{1-\frac{1}{\rho}} - \frac{1-\frac{1}{\rho}}{1-\frac{1}{\rho}}} (1-\frac{1}{\rho})c_t^{-\frac{1}{\rho}}\\
		&= \left[c_t^{1-\frac{1}{\rho}} + \delta c_{t+1}^{1-\frac{1}{\rho}}\right]^{\frac{-\gamma + \frac{1}{\rho}}{1-\frac{1}{\rho}}} c_t^{-\frac{1}{\rho}}\\
		u_{c_{t+1}}' &= \frac{\partial u}{\partial c_{t+1}} = \frac{1}{1-1\rho}\left[c_t^{1-\frac{1}{\rho}} + \delta c_{t+1}^{1-\frac{1}{\rho}}\right]^{\frac{1-\gamma}{1-\frac{1}{\rho}} - \frac{1-\frac{1}{\rho}}{1-\frac{1}{\rho}}} \delta (1-\frac{1}{\rho})c_{t+1}^{-\frac{1}{\rho}}\\
		&= \left[c_t^{1-\frac{1}{\rho}} + \delta c_{t+1}^{1-\frac{1}{\rho}}\right]^{\frac{-\gamma + \frac{1}{\rho}}{1-\frac{1}{\rho}}} \delta c_{t+1}^{-\frac{1}{\rho}}\\
		\text{IMRS} &= \left( \frac{u_{c_{t+1}}'}{u_{c_t}'} \right) = \delta \left(\frac{c_{t+1}}{c_t}\right)^{-\frac{1}{\rho}}\\
	\end{align*}
	
	Again, the optimal portfolio found forming the Lagrangian and FOC
	\begin{align*}
		L &= U(c_t, c_{t+1})  - \phi \left(c_t + \frac{c_{t+1}}{1+R} - y_t - \frac{y_{t+1}}{1+R}\right)\\
		0 &= u_{c_t}' - \phi\\
		0 & = u_{c_{t+1}}' - \phi (1+R)^{-1}\\
		\text{IMRS} &= \left( \frac{u_{c_{t+1}}'}{u_{c_t}'} \right) = \delta \left(\frac{c_{t+1}}{c_t}\right)^{-\frac{1}{\rho}} = (1+R)^{-1}\\
	\end{align*}

	The IES is derived 
	\begin{align*}
		\delta \left(\frac{c_{t+1}}{c_t}\right)^{-\frac{1}{\rho}} &= (1+R)^{-1}\\
		\left(\frac{c_{t+1}}{c_t}\right)^{-\frac{1}{\rho}} &= \delta^{-1}(1+R)^{-1}\\
		\left(\frac{c_{t+1}}{c_t}\right) &= \delta^{\rho}(1+R)^{\rho}\\
		\ln \left(\frac{c_{t+1}}{c_t}\right) &= \rho \ln \delta + \rho \ln (1+R)\\
		\text{IES} &= \frac{\partial \ln \left(\frac{c_{t+1}}{c_t}\right)}{\partial \ln (1+R)} = \rho\\
	\end{align*}

	Again, the optimal portfolio is determined by the agents choice of $c_t$. Given that $\delta$, $\gamma = \frac{1}{\rho}$, the endowment $w_0 = y_t + \frac{y_{t+1}}{1+R} = c_t + \frac{c_{t+1}}{1+R}$ and R are known and certain at time t, a choice of $c_t$ implies a choice of $c_{t+1}$ and thereby the ratio $\frac{c_{t+1}}{c_t}$. The optimal portfolio choice then will be such that $$\frac{c_{t+1}}{c_t} = \delta^{\rho}(1+R)^{\rho} =  \delta^{\frac{1}{\gamma}}(1+R)^{\frac{1}{\gamma}}$$
	
	The sensitivity of the optimal portfolio to changes in $\gamma$ and the IES $\rho = \frac{1}{\gamma}$
	\begin{align*}
		\ln \left(\frac{c_{t+1}}{c_t}\right) &= \rho \ln \delta + \rho \ln (1+R)\\
		\frac{\partial \ln \left(\frac{c_{t+1}}{c_t}\right)}{\partial \rho} &= \ln \delta + \ln (1+R)\\
	\end{align*}
	
	A decrease in risk aversion $\gamma$, i.e. an increase of the IES $\rho = \frac{1}{\gamma}$ increases the preference for future consumption by a percentage equal to the sum of logs of personal time preference $\delta$ and gross return 1+R. Given constant personal time preference $\delta$, the less risk averse agent accepts to reduce current consumption and invests more in the tradable asset when she is compensated by a corresponding increase in the rate of return. 
	
	%\section{Annex 1}
	
\end{document}