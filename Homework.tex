\documentclass[]{article}
\usepackage{amsmath}
\usepackage{amssymb}
\usepackage{easyReview}

%opening
\title{FIN800 Discrete Time Financial Economics - Homework 1}
\author{Giovanni Maffei, Timo Predöhl}

\begin{document}

\maketitle

\section{portfolio choice}
An agent with initial wealth $w_t$ at time t invests a fraction of wealth $\frac{\phi}{w_t}$ in the risky asset and the remainder $(1-\frac{\phi}{w_t})$ in the riskless asset. The resulting wealth at time t+1 is:
\begin{align*}
	R_{f,t+1} &= R_{f} = 1+r_f = const.\\
	R_{t+1} &= 1 + \mu + \beta x_t + \sigma_r \eta_{r, t+1}\\
	w_{t+1} &= w_t[(1-\frac{\phi}{w_t})R_f+\frac{\phi}{w_t} R_{t+1}]\\
	&= w_t[R_f - \frac{\phi}{w_t} R_f + \frac{\phi}{w_t} R_{t+1}]\\
	&= w_t[R_f + \frac{\phi}{w_t} (R_{t+1} - R_f)]\\
	&= w_t R_f + \phi (R_{t+1} - R_f)
\end{align*}

The agent's utility function then is:
\begin{align*}
	u(w_{t+1}) &= -e^{-aw_{t+1} - bESG_t}\\
	&= -e^{-a(w_t R_f + \phi (R_{t+1} - R_f)) - bESG_t}\\
\end{align*}

The random variable $R_{t+1}$ is normally distributed with 
\begin{align*}
	E[R_{t+1}] &= 1+\mu+\beta x_t\\
	V[R_{t+1}] &= E[(R_{t+1} - E[R_{t+1}])^2]\\
	&= E[(1 + \mu + \beta x_t + \sigma_r \eta_{r, t+1} - 1 - \mu - \beta x_t)^2]\\
	&= E[\sigma_r^2 \eta_{r, t+1}^2]\\
	&= \sigma_r^2 E[\eta_{r, t+1}^2]\\
	&= \sigma_r^2
\end{align*}

The random variable $\tilde{x} = aw_{t+1} + bESG_t$ is normally distributed with 
\begin{align*}
	E[\tilde{x}] &= aE[w_t R_f + \phi (R_{t+1} - R_f)] + bE[ESG_t]\\
	&= a(w_t R_f + \phi (E[R_{t+1}] - R_f)) + bESG_t\\
	V[\tilde{x}] &= E[(\tilde{x} - E[\tilde{x}])^2]\\
	&= E[(a(w_t R_f + \phi (R_{t+1} - R_f)) + bESG_t \\
	&- a(w_t R_f + \phi (E[R_{t+1}] - R_f)) - bESG_t)^2]\\
	&= E[(a\phi (R_{t+1}) - a \phi E[R_{t+1}])^2]\\
	&= a^2 \phi^2 E[(R_{t+1} - E[R_{t+1}])^2]\\
	&= a^2 \phi^2 \sigma_r^2\\
\end{align*}

Therefore, using the fact that the expectation of the exponential of a normally distributed random variable is the exponential of the mean plus half of the variance, the expected utility is
\begin{align*}
	E[u(w_{t+1})] &= E[-e^{-a(w_t R_f + \phi (R_{t+1} - R_f)) - bESG_t}]\\
	&= -e^{-aw_t R_f -a \phi (1+\mu+\beta x_t - R_f) - bESG_t + \frac{1}{2}a^2 \phi^2 \sigma_r^2}
\end{align*}

Maximizing expected utility is equivalent to maximizing the certainty equivalent $$-aw_t R_f -a \phi (E[R_{t+1}] - R_f) - bESG_t + \frac{1}{2}a^2 \phi^2 \sigma_r^2$$.

Differentiating with respect to $\phi$ and setting the derivative to zero yields: 
\begin{align*}
	0 &= \frac{\partial (-aw_t R_f -a \phi (E[R_{t+1}] - R_f) - bESG_t + \frac{1}{2}a^2 \phi^2 \sigma_r^2)}{\partial(\phi)}\\
	&= -a (E[R_{t+1}] - R_f) + \phi a^2 \sigma_r^2\\
	\phi &= \frac{a (E[R_{t+1}] - R_f)}{a^2 \sigma_r^2}\\
	&= \frac{(E[R_{t+1}] - R_f)}{a \sigma_r^2}\\
	&= \frac{1}{a \sigma_r}\text{Sharpe}
\end{align*}

\subsection{What will be the optimal portfolio choice of the investor ?}
The optimal amount $\phi$ to invest in the risky asset is $$\phi^* = \frac{(E[R_{t+1}] - R_f)}{a \sigma_r^2} = \frac{1}{a \sigma_r}\text{Sharpe}$$.

The optimal choice $\phi^*$ 
\begin{itemize}
	\item is an increasing function of the risk premium $E[R_{t+1}]-R_f$,
	\item is a decreasing function of the variance $\sigma_r^2$,
	\item is a decreasing function of the investor’s absolute risk aversion a,
	\item is positive when the risk premium is positive, and
	\item does not depend on the initial wealth $w_t$.
\end{itemize}.

\subsection{What will be the mean and variance of the corresponding optimal wealth?}
Optimal wealth is obtained by inserting the optimal portfolio into the wealth function:
\begin{align*}
	E[w_{t+1}] &= w_t R_f + \phi^* (E[R_{t+1}]-R_f)\\
	&= w_t R_f + \frac{(E[R_{t+1}] - R_f)^2}{a \sigma_r^2}\\
	&= w_t R_f + \frac{1}{a}\text{Sharpe}^2\\
	V[w_{t+1}] &= E[(w_{t+1} - E[w_{t+1}])^2]\\
	&= E[(w_t R_f + \phi^* (R_{t+1}-R_f) - w_t R_f - \phi^* (E[R_{t+1}]-R_f))^2]\\
	&= E[(\phi^* R_{t+1} - \phi^* R_f - \phi^* E[R_{t+1}] + \phi^* R_f)^2]\\
	&= E[(\phi^*(R_{t+1} - E[R_{t+1}]))^2]\\
	&= \phi^{*2} V[R_{t+1}]\\
	&= \frac{(E[R_{t+1}] - R_f)^2}{a^2 \sigma_r^4} \sigma_r^2\\
	&= \frac{1}{a^2}\text{Sharpe}^2
\end{align*}

\subsection{Compute the welfare of the investor.}
The welfare of the investor is obtained by inserting the optimal portfolio $\phi^*$ into certainty equivalent: 
\begin{align*}
	\text{welfare} &= -aw_t R_f -a \phi^* (E[R_{t+1}] - R_f) - bESG_t + \frac{1}{2}a^2 \phi^{*2} \sigma_r^2\\
	&= -aw_t R_f -a \frac{1}{a \sigma_r}\text{Sharpe} (E[R_{t+1}] - R_f) - bESG_t + \frac{1}{2}a^2 \frac{1}{a^2 \sigma_r^2}\text{Sharpe}^2 \sigma_r^2\\
	&= -aw_t R_f -\text{Sharpe}^2 - bESG_t + \frac{1}{2}\text{Sharpe}^2\\
	&= -aw_t R_f - bESG_t - \frac{1}{2}\text{Sharpe}^2\\
\end{align*}

\subsection{What is the distribution of the optimal wealth ? Comment.}

Expected optimal wealth $$E_t[w_{t+1}] = w_t R_f + \frac{1}{a}\text{Sharpe}^2$$ is a function of the random variable $R_{t+1}$. Its components $x_t$ and $\sigma_r$ are determined at time t, i.e. are no longer random. Randomness remains in the form of the normally distributed $\eta_{r,t+1}$. The expected optimal wealth therefore is \comment{normally distributed}{Here, I don't really have an idea. It appears to be too simplistic as the absence of randomness at t does not necessarily reduce to normality.} as well.

\subsection{Comment on the impact of preference for ESG on the optimal portfolio.}
Preferences for ESG enter the agent's utility function with factor b. The factor $ESG_t$ represents the ESG rating of the risky asset at time t. Consequently, a 1-notch better ESG rating raises absolut utility by b. The motion of wealth, the return on the risky asset and therefore the optimal portfolio choice are unaffected by the ESG rating.

\subsection{Now assume that the ESG rating is uncertain and follows a normal distribution ($\mu, \sigma$), independent from $\eta_{r,t+1}$. Repeat the previous questions.}
The random variable $\tilde{x} = aw_{t+1} + bESG_t$ is normally distributed with 
\begin{align*}
	E[\tilde{x}] &= aE[w_t R_f + \phi (R_{t+1} - R_f)] + bE[ESG_t]\\
	&= a(w_t R_f + \phi (E[R_{t+1}] - R_f)) + bE[ESG_t]\\
	V[\tilde{x}] &= E[(\tilde{x} - E[\tilde{x}])^2]\\
	&= E[(a(w_t R_f + \phi (R_{t+1} - R_f)) + bESG_t \\
	&- a(w_t R_f + \phi (E[R_{t+1}] - R_f)) - bE[ESG_t])^2]\\
	&= E[(a\phi (R_{t+1} - E[R_{t+1}]) +b(ESG_t - E[ESG_t]))^2]\\
	&= a^2 \phi^2 \sigma_r^2 + b^2 \sigma_{esg}^2 + E[2 a\phi (R_{t+1} - E[R_{t+1}]) b(ESG_t - E[ESG_t])]\\
	&= a^2 \phi^2 \sigma_r^2 + b^2 \sigma_{esg}^2 + E[2 a\phi \sigma_r \eta_{r, t+1} b(ESG_t - E[ESG_t])]\\
	&= a^2 \phi^2 \sigma_r^2 + b^2 \sigma_{esg}^2 \text{, given independence of ESG and $\eta_{r, t+1}$}\\
\end{align*}


\section{inter-temporal consumption}
The 2-period, single asset utility maximization program is formulated as
\begin{align*}
	\max_{c_t, c_{t+1}} U(c_t, c_{t+1}) &= u(c_t) + \delta u(c_{t+1}), s.t. c_t = y_t - \lambda, c_{t+1} = y_{t+1} + \lambda (1+R)\\
	&= \frac{c_t^{1-\gamma}}{1-\gamma} + \delta \frac{c_{t+1}^{1-\gamma}}{1-\gamma}, s.t. c_t + \frac{c_{t+1}}{1+R} = y_t + \frac{y_{t+1}}{1+R}
\end{align*}

Formulating the Lagrangian and FOCs for $c_t$ and $c_{t+1}$ yields
\begin{align}
	L &= u(c_t) + \delta u(c_{t+1}) - \phi (\frac{c_{t+1}}{1+R} - y_t - \frac{y_{t+1}}{1+R})\\
	\frac{\partial L}{\partial c_t} &= u_{c_t}' - \phi = 0\\
	\frac{\partial L}{\partial c_{t+1}} &= \delta u_{c_{t+1}}' - \phi \frac{1}{1+R}= 0\\
	\frac{\delta u_{c_{t+1}}'}{u_{c_t}'} &= \frac{1}{1+R}\\
	\delta \left( \frac{c_{t+1}}{c_t} \right)^{-\gamma} &= \frac{1}{1+R} = \text{IMRS}\\
	\frac{c_{t+1}}{c_t} &= \delta^{\frac{1}{\gamma}}(1+R)^{\frac{1}{\gamma}}\\
	\frac{\partial(\frac{c_{t+1}}{c_t})}{\partial(1+R)} \frac{(1+R)}{\frac{c_{t+1}}{c_t}} &= \frac{1}{\gamma} = \text{IES}
\end{align}

\subsection{What will be the optimal portfolio and how sensitive it is relative to $\gamma$?}
Given known endowment $w_0 = y_t + \frac{y_{t+1}}{1+R}$, known return on the risky asset R and consequently the maximum achievable utility $\bar{U}$, the optimal portfolio is determined by the IMRS (5), whereby the agent would at most forego current consumption in the amount of the IMRS in order invest this amount, earn R and consume an additional unit in the future.  The optimal portfolio and therefore the ratio of future consumption to current consumption is sensitiv to $\gamma$
\begin{align*}
	\left(\frac{c_{t+1}}{c_t}\right)^\gamma &= \delta (1+R)\\
	\gamma \ln \left(\frac{c_{t+1}}{c_t}\right) &= \ln \delta + \ln (1+R)\\
\end{align*}

\subsection{Rewrite the utility function in a recursive way using a CES aggregator. What will then be the optimal portfolio and how sensitive it is relative to $\gamma$ and the IES ?}
The original utility function is turned into a CES aggregator using:
\begin{align*}
	U(c_t, c_{t+1}) &= \frac{1}{1-\gamma} \left[c_t^{1-1/\rho} + \delta c_{t+1}^{1-1/\rho}\right]^{\frac{1-\gamma}{1-1/\rho}}\\
	IMRS &: \left( \frac{u_{c_{t+1}}'}{u_{c_t}'} \right) = \delta \left(\frac{c_{t+1}}{c_t}\right)^{-1/\rho}\\
	-\ln IMRS &= \ln \delta -\frac{1}{\rho} \ln \left(\frac{c_{t+1}}{c_t}\right)\\
	\rho \ln IMRS &= -\rho \ln \delta + \ln \left(\frac{c_{t+1}}{c_t}\right)\\
	IES &: \frac{\partial \ln \left(\frac{c_{t+1}}{c_t}\right)}{\partial \ln IMRS} = \rho\\
\end{align*}

\end{document}